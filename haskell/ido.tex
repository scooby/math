\documentclass{article}

%% ODER: format ==         = "\mathrel{==}"
%% ODER: format /=         = "\neq "
%
%
\makeatletter
\@ifundefined{lhs2tex.lhs2tex.sty.read}%
  {\@namedef{lhs2tex.lhs2tex.sty.read}{}%
   \newcommand\SkipToFmtEnd{}%
   \newcommand\EndFmtInput{}%
   \long\def\SkipToFmtEnd#1\EndFmtInput{}%
  }\SkipToFmtEnd

\newcommand\ReadOnlyOnce[1]{\@ifundefined{#1}{\@namedef{#1}{}}\SkipToFmtEnd}
\usepackage{amstext}
\usepackage{amssymb}
\usepackage{stmaryrd}
\DeclareFontFamily{OT1}{cmtex}{}
\DeclareFontShape{OT1}{cmtex}{m}{n}
  {<5><6><7><8>cmtex8
   <9>cmtex9
   <10><10.95><12><14.4><17.28><20.74><24.88>cmtex10}{}
\DeclareFontShape{OT1}{cmtex}{m}{it}
  {<-> ssub * cmtt/m/it}{}
\newcommand{\texfamily}{\fontfamily{cmtex}\selectfont}
\DeclareFontShape{OT1}{cmtt}{bx}{n}
  {<5><6><7><8>cmtt8
   <9>cmbtt9
   <10><10.95><12><14.4><17.28><20.74><24.88>cmbtt10}{}
\DeclareFontShape{OT1}{cmtex}{bx}{n}
  {<-> ssub * cmtt/bx/n}{}
\newcommand{\tex}[1]{\text{\texfamily#1}}	% NEU

\newcommand{\Sp}{\hskip.33334em\relax}


\newcommand{\Conid}[1]{\mathit{#1}}
\newcommand{\Varid}[1]{\mathit{#1}}
\newcommand{\anonymous}{\kern0.06em \vbox{\hrule\@width.5em}}
\newcommand{\plus}{\mathbin{+\!\!\!+}}
\newcommand{\bind}{\mathbin{>\!\!\!>\mkern-6.7mu=}}
\newcommand{\rbind}{\mathbin{=\mkern-6.7mu<\!\!\!<}}% suggested by Neil Mitchell
\newcommand{\sequ}{\mathbin{>\!\!\!>}}
\renewcommand{\leq}{\leqslant}
\renewcommand{\geq}{\geqslant}
\usepackage{polytable}

%mathindent has to be defined
\@ifundefined{mathindent}%
  {\newdimen\mathindent\mathindent\leftmargini}%
  {}%

\def\resethooks{%
  \global\let\SaveRestoreHook\empty
  \global\let\ColumnHook\empty}
\newcommand*{\savecolumns}[1][default]%
  {\g@addto@macro\SaveRestoreHook{\savecolumns[#1]}}
\newcommand*{\restorecolumns}[1][default]%
  {\g@addto@macro\SaveRestoreHook{\restorecolumns[#1]}}
\newcommand*{\aligncolumn}[2]%
  {\g@addto@macro\ColumnHook{\column{#1}{#2}}}

\resethooks

\newcommand{\onelinecommentchars}{\quad-{}- }
\newcommand{\commentbeginchars}{\enskip\{-}
\newcommand{\commentendchars}{-\}\enskip}

\newcommand{\visiblecomments}{%
  \let\onelinecomment=\onelinecommentchars
  \let\commentbegin=\commentbeginchars
  \let\commentend=\commentendchars}

\newcommand{\invisiblecomments}{%
  \let\onelinecomment=\empty
  \let\commentbegin=\empty
  \let\commentend=\empty}

\visiblecomments

\newlength{\blanklineskip}
\setlength{\blanklineskip}{0.66084ex}

\newcommand{\hsindent}[1]{\quad}% default is fixed indentation
\let\hspre\empty
\let\hspost\empty
\newcommand{\NB}{\textbf{NB}}
\newcommand{\Todo}[1]{$\langle$\textbf{To do:}~#1$\rangle$}

\EndFmtInput
\makeatother
%
%
%
%
%
%
% This package provides two environments suitable to take the place
% of hscode, called "plainhscode" and "arrayhscode". 
%
% The plain environment surrounds each code block by vertical space,
% and it uses \abovedisplayskip and \belowdisplayskip to get spacing
% similar to formulas. Note that if these dimensions are changed,
% the spacing around displayed math formulas changes as well.
% All code is indented using \leftskip.
%
% Changed 19.08.2004 to reflect changes in colorcode. Should work with
% CodeGroup.sty.
%
\ReadOnlyOnce{polycode.fmt}%
\makeatletter

\newcommand{\hsnewpar}[1]%
  {{\parskip=0pt\parindent=0pt\par\vskip #1\noindent}}

% can be used, for instance, to redefine the code size, by setting the
% command to \small or something alike
\newcommand{\hscodestyle}{}

% The command \sethscode can be used to switch the code formatting
% behaviour by mapping the hscode environment in the subst directive
% to a new LaTeX environment.

\newcommand{\sethscode}[1]%
  {\expandafter\let\expandafter\hscode\csname #1\endcsname
   \expandafter\let\expandafter\endhscode\csname end#1\endcsname}

% "compatibility" mode restores the non-polycode.fmt layout.

\newenvironment{compathscode}%
  {\par\noindent
   \advance\leftskip\mathindent
   \hscodestyle
   \let\\=\@normalcr
   \let\hspre\(\let\hspost\)%
   \pboxed}%
  {\endpboxed\)%
   \par\noindent
   \ignorespacesafterend}

\newcommand{\compaths}{\sethscode{compathscode}}

% "plain" mode is the proposed default.
% It should now work with \centering.
% This required some changes. The old version
% is still available for reference as oldplainhscode.

\newenvironment{plainhscode}%
  {\hsnewpar\abovedisplayskip
   \advance\leftskip\mathindent
   \hscodestyle
   \let\hspre\(\let\hspost\)%
   \pboxed}%
  {\endpboxed%
   \hsnewpar\belowdisplayskip
   \ignorespacesafterend}

\newenvironment{oldplainhscode}%
  {\hsnewpar\abovedisplayskip
   \advance\leftskip\mathindent
   \hscodestyle
   \let\\=\@normalcr
   \(\pboxed}%
  {\endpboxed\)%
   \hsnewpar\belowdisplayskip
   \ignorespacesafterend}

% Here, we make plainhscode the default environment.

\newcommand{\plainhs}{\sethscode{plainhscode}}
\newcommand{\oldplainhs}{\sethscode{oldplainhscode}}
\plainhs

% The arrayhscode is like plain, but makes use of polytable's
% parray environment which disallows page breaks in code blocks.

\newenvironment{arrayhscode}%
  {\hsnewpar\abovedisplayskip
   \advance\leftskip\mathindent
   \hscodestyle
   \let\\=\@normalcr
   \(\parray}%
  {\endparray\)%
   \hsnewpar\belowdisplayskip
   \ignorespacesafterend}

\newcommand{\arrayhs}{\sethscode{arrayhscode}}

% The mathhscode environment also makes use of polytable's parray 
% environment. It is supposed to be used only inside math mode 
% (I used it to typeset the type rules in my thesis).

\newenvironment{mathhscode}%
  {\parray}{\endparray}

\newcommand{\mathhs}{\sethscode{mathhscode}}

% texths is similar to mathhs, but works in text mode.

\newenvironment{texthscode}%
  {\(\parray}{\endparray\)}

\newcommand{\texths}{\sethscode{texthscode}}

% The framed environment places code in a framed box.

\def\codeframewidth{\arrayrulewidth}
\RequirePackage{calc}

\newenvironment{framedhscode}%
  {\parskip=\abovedisplayskip\par\noindent
   \hscodestyle
   \arrayrulewidth=\codeframewidth
   \tabular{@{}|p{\linewidth-2\arraycolsep-2\arrayrulewidth-2pt}|@{}}%
   \hline\framedhslinecorrect\\{-1.5ex}%
   \let\endoflinesave=\\
   \let\\=\@normalcr
   \(\pboxed}%
  {\endpboxed\)%
   \framedhslinecorrect\endoflinesave{.5ex}\hline
   \endtabular
   \parskip=\belowdisplayskip\par\noindent
   \ignorespacesafterend}

\newcommand{\framedhslinecorrect}[2]%
  {#1[#2]}

\newcommand{\framedhs}{\sethscode{framedhscode}}

% The inlinehscode environment is an experimental environment
% that can be used to typeset displayed code inline.

\newenvironment{inlinehscode}%
  {\(\def\column##1##2{}%
   \let\>\undefined\let\<\undefined\let\\\undefined
   \newcommand\>[1][]{}\newcommand\<[1][]{}\newcommand\\[1][]{}%
   \def\fromto##1##2##3{##3}%
   \def\nextline{}}{\) }%

\newcommand{\inlinehs}{\sethscode{inlinehscode}}

% The joincode environment is a separate environment that
% can be used to surround and thereby connect multiple code
% blocks.

\newenvironment{joincode}%
  {\let\orighscode=\hscode
   \let\origendhscode=\endhscode
   \def\endhscode{\def\hscode{\endgroup\def\@currenvir{hscode}\\}\begingroup}
   %\let\SaveRestoreHook=\empty
   %\let\ColumnHook=\empty
   %\let\resethooks=\empty
   \orighscode\def\hscode{\endgroup\def\@currenvir{hscode}}}%
  {\origendhscode
   \global\let\hscode=\orighscode
   \global\let\endhscode=\origendhscode}%

\makeatother
\EndFmtInput
%
\usepackage{amsmath}

\begin{document}

\section{Name}

Ido base $(-1 + i)$ conversion 

\subsection{Description}

ido (short for $i$ dash one, as meaning $i - 1$) is a
library to convert a pair of integers to a unique whole number using a
complex base conversion.

\subsection{Conjecture}

I propose that any value $a + bi$ where $a$ and $b$ are integers can be
represented as a binary string $c$ of $n$ bits such that:

\[
  a + bi = c_0*(-1+i)^0 + c_1*(-1+i)^1 + \ldots{} + c_n*(-1+i)^n
\]

$c$ itself is also an integer in base 2 form.
Further, although $a$ and $b$ are integers, $c$ is always a whole number.

\begin{hscode}\SaveRestoreHook
\column{B}{@{}>{\hspre}l<{\hspost}@{}}%
\column{3}{@{}>{\hspre}l<{\hspost}@{}}%
\column{8}{@{}>{\hspre}l<{\hspost}@{}}%
\column{9}{@{}>{\hspre}l<{\hspost}@{}}%
\column{E}{@{}>{\hspre}l<{\hspost}@{}}%
\>[B]{}\mathbf{module}\;\Conid{\Conid{Base}.Ido}\;(\Varid{ido2abi},\Varid{abi2ido}){}\<[E]%
\\
\>[B]{}\hsindent{8}{}\<[8]%
\>[8]{}\mathbf{where}{}\<[E]%
\\[\blanklineskip]%
\>[B]{}\mathbf{import}\;\Conid{\Conid{Data}.Bits}{}\<[E]%
\\[\blanklineskip]%
\>[B]{}\mbox{\onelinecomment  \ensuremath{\Conid{Fold}\;\Varid{over}\;\Varid{bits},\Varid{equals}\mathbin{:}}}{}\<[E]%
\\
\>[B]{}\mbox{\onelinecomment  (v[n] `f` ... (v[2] `f` (v[1] `f` (v[0] `f` z))) ... )}{}\<[E]%
\\
\>[B]{}\mbox{\onelinecomment }{}\<[E]%
\\
\>[B]{}\mbox{\onelinecomment  Note: if v is -1, foldr will not terminate}{}\<[E]%
\\
\>[B]{}\Varid{bitFoldr}\mathbin{::}\Conid{Bits}\;\Varid{e}\Rightarrow (\Varid{e}\to \Varid{a}\to \Varid{a})\to \Varid{a}\to \Varid{e}\to \Varid{a}{}\<[E]%
\\
\>[B]{}\Varid{bitFoldr}\;\Varid{f}\;\Varid{z}\;\Varid{v}\mathrel{=}\Varid{go}\;\Varid{v}{}\<[E]%
\\
\>[B]{}\hsindent{3}{}\<[3]%
\>[3]{}\mathbf{where}\;\Varid{go}\;\mathrm{0}\mathrel{=}\Varid{z}{}\<[E]%
\\
\>[3]{}\hsindent{6}{}\<[9]%
\>[9]{}\Varid{go}\;\Varid{v}\mathrel{=}(\Varid{v}\mathbin{.\&.}\mathrm{1})\mathbin{`\Varid{f}`}\Varid{go}\;(\Varid{v}\mathbin{`\Varid{shiftR}`}\mathrm{1}){}\<[E]%
\ColumnHook
\end{hscode}\resethooks

\subsection{ido2abi}

Operates as your standard base 2 conversion, only instead of each place
representing a power of 2 it represents a power of $(-1 + i)$.

At the least significant bit, $(-1 + i)^0 = (1 + 0i)$.

To find the next highest power of $(-1 + i)$, the formula is a standard
complex multiplication:

\begin{align}
  (a + bi) * (-1 + i) &= -a + ai - bi - b \\
                      &= (-a - b) + (a - b)i
\end{align}

\begin{hscode}\SaveRestoreHook
\column{B}{@{}>{\hspre}l<{\hspost}@{}}%
\column{3}{@{}>{\hspre}l<{\hspost}@{}}%
\column{11}{@{}>{\hspre}l<{\hspost}@{}}%
\column{E}{@{}>{\hspre}l<{\hspost}@{}}%
\>[B]{}\mbox{\onelinecomment  \ensuremath{\Conid{Given}\;\Varid{a}\;\Varid{whole}\;\Varid{integer}\;\Varid{c},\Varid{return}\;\Varid{a}\mathbin{+}\Varid{bi}}}{}\<[E]%
\\
\>[B]{}\Varid{ido2abi}\mathbin{::}\Conid{Bits}\;\Varid{a}\Rightarrow \Varid{a}\to (\Varid{a},\Varid{a}){}\<[E]%
\\
\>[B]{}\Varid{ido2abi}\;\Varid{i}\mathrel{=}\Varid{fst}\mathbin{\$}\Varid{bitFoldr}\;\Varid{f}\;((\mathrm{0},\mathrm{0}),(\mathrm{1},\mathrm{0}))\;\Varid{i}{}\<[E]%
\\
\>[B]{}\hsindent{3}{}\<[3]%
\>[3]{}\mathbf{where}\;\Varid{f}\;\Varid{e}\;((\Varid{sa},\Varid{sb}),(\Varid{a},\Varid{b}))\mathrel{=}{}\<[E]%
\\
\>[3]{}\hsindent{8}{}\<[11]%
\>[11]{}((\Varid{sa}\mathbin{+}\Varid{a}\mathbin{*}\Varid{e},\Varid{sb}\mathbin{+}\Varid{b}\mathbin{*}\Varid{e}),((\mathbin{-}\Varid{a})\mathbin{-}\Varid{b},\Varid{a}\mathbin{-}\Varid{b})){}\<[E]%
\ColumnHook
\end{hscode}\resethooks
          
\subsection{abi2ido}

Given the integers $a$ and $b$, returns a unique integer $c$.

This algorithm works by repeatedly dividing $a + bi$ by $(-1 + i)$.

Within the loop, we check to see if the number is odd. If $a + b$ is odd,
we know the least significant bit is $1$. And we can always subtract by
one by subtracting one from $a$.

That guarantees us that the value is divisible by $(-1 + i)$. Division is
by multiplication by $(i + 1) / (i + 1)$:

\[
\frac{a+bi}{-1+i} = \frac{(a+bi)*(1+i)}{(-1+i)*(1+i)} = 
\frac{a+ai+bi-b}{-1-1} = \frac{a-b}{-2}+\frac{(a+b)i}{-2}
\]

\begin{hscode}\SaveRestoreHook
\column{B}{@{}>{\hspre}l<{\hspost}@{}}%
\column{3}{@{}>{\hspre}l<{\hspost}@{}}%
\column{11}{@{}>{\hspre}l<{\hspost}@{}}%
\column{17}{@{}>{\hspre}l<{\hspost}@{}}%
\column{19}{@{}>{\hspre}l<{\hspost}@{}}%
\column{E}{@{}>{\hspre}l<{\hspost}@{}}%
\>[B]{}\Varid{abi2ido}\mathbin{::}\Conid{Bits}\;\Varid{a}\Rightarrow (\Varid{a},\Varid{a})\to \Varid{a}{}\<[E]%
\\
\>[B]{}\Varid{abi2ido}\;(\Varid{a},\Varid{b})\mathrel{=}\Varid{f}\;\mathrm{1}\;\mathrm{0}\;\Varid{a}\;\Varid{b}{}\<[E]%
\\
\>[B]{}\hsindent{3}{}\<[3]%
\>[3]{}\mathbf{where}\;\Varid{f}\;\Varid{x}\;\Varid{i}\;\Varid{a}\;\Varid{b}\mid \Varid{a'}\equiv \mathrm{0}\mathrel{\wedge}\Varid{b'}\equiv \mathrm{0}\mathrel{=}\Varid{i'}{}\<[E]%
\\
\>[3]{}\hsindent{16}{}\<[19]%
\>[19]{}\mid \Varid{otherwise}\mathrel{=}\Varid{f}\;\Varid{x'}\;\Varid{i'}\;\Varid{a''}\;\Varid{b'}{}\<[E]%
\\
\>[3]{}\hsindent{8}{}\<[11]%
\>[11]{}\mathbf{where}\;\Varid{odd}\mathrel{=}(\Varid{a}\mathbin{+}\Varid{b})\mathbin{.\&.}\mathrm{1}{}\<[E]%
\\
\>[11]{}\hsindent{6}{}\<[17]%
\>[17]{}\Varid{a'}\mathrel{=}\Varid{a}\mathbin{-}\Varid{odd}{}\<[E]%
\\
\>[11]{}\hsindent{6}{}\<[17]%
\>[17]{}\Varid{i'}\mathrel{=}\Varid{i}\mathbin{+}\Varid{x}\mathbin{*}\Varid{odd}{}\<[E]%
\\
\>[11]{}\hsindent{6}{}\<[17]%
\>[17]{}\Varid{a''}\mathrel{=}\Varid{divn2}\mathbin{\$}\Varid{a'}\mathbin{-}\Varid{b}{}\<[E]%
\\
\>[11]{}\hsindent{6}{}\<[17]%
\>[17]{}\Varid{b'}\mathrel{=}\Varid{divn2}\mathbin{\$}\Varid{a'}\mathbin{+}\Varid{b}{}\<[E]%
\\
\>[11]{}\hsindent{6}{}\<[17]%
\>[17]{}\Varid{x'}\mathrel{=}\Varid{shiftL}\;\Varid{x}\;\mathrm{1}{}\<[E]%
\\
\>[11]{}\hsindent{6}{}\<[17]%
\>[17]{}\Varid{divn2}\mathrel{=}(\mathbin{`\Varid{shiftR}`}\mathrm{1})\mathbin{\circ}\Varid{negate}{}\<[E]%
\ColumnHook
\end{hscode}\resethooks

\end{document}
